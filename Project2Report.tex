\documentclass[11pt]{article}

%  USE PACKAGES  ---------------------- 
\usepackage[margin=0.75in,vmargin=1in]{geometry}
\usepackage{amsmath,amsthm,amsfonts}
\usepackage{amssymb}
\usepackage{fancyhdr}
\usepackage{enumerate}
\usepackage{mathtools}
\usepackage{hyperref,color}
\usepackage{enumitem,amssymb}
\newlist{todolist}{itemize}{4}
\setlist[todolist]{label=$\square$}
\usepackage{pifont}
\newcommand{\cmark}{\ding{51}}%
\newcommand{\xmark}{\ding{55}}%
\newcommand{\done}{\rlap{$\square$}{\raisebox{2pt}{\large\hspace{1pt}\cmark}}%
\hspace{-2.5pt}}
\newcommand{\HREF}[2]{\href{#1}{#2}}
\usepackage{textcomp}
\usepackage{listings}
\lstset{
basicstyle=\small\ttfamily,
% columns=flexible,
upquote=true,
breaklines=true,
showstringspaces=false
}

\renewcommand\paragraph{{\normalfont\normalsize}}
%  -------------------------------------------- 
%https://www.overleaf.com/project/63cab2b14d247bee612dcc29https://www.overleaf.com/project/63cab2b14d247bee612dcc29
%  HEADER AND FOOTER (DO NOT EDIT) ----------------------
\newcommand{\problemnumber}{0}
\pagestyle{fancy}
\fancyhead{}
\fancyhead[L]{\textbf{\sectionheader}}
\newcommand{\newquestion}[1]{
\clearpage % page break and flush floats
\renewcommand{\problemnumber}{#1} % set problem number for header
\phantom{}  % Put something on the page so it shows
}
\fancyfoot[L]{IE 332}
\fancyfoot[C]{Assignment submission}
\fancyfoot[R]{Page \thepage}
\renewcommand{\footrulewidth}{0.4pt}

%  --------------------------------------------


%  COVER SHEET (FILL IN THE TABLE AS INSTRUCTED IN THE ASSIGNMENT) ----------------------
\newcommand{\addcoversheet}{
\clearpage
\thispagestyle{empty}
\vspace*{0.5in}

\begin{center}
\Huge{{\bf IE332 Project 2}} % <-- replace with correct assignment #

Due: April 28th, 11:59pm EST % <-- replace with correct due date and time
\end{center}

\vspace{0.3in}

\noindent We have {\bf read and understood the assignment instructions}. We certify that the submitted work does not violate any academic misconduct rules, and that it is solely our own work. By listing our names below we acknowledge that any misconduct will result in appropriate consequences. 

\vspace{0.2in}

\noindent {\em ``As a Boilermaker pursuing academic excellence, I pledge to be honest and true in all that I do.
Accountable together -- we are Purdue.''}

\vspace{0.3in}

\begin{table}[h!]
  \begin{center}
    \label{tab:table1}
    \begin{tabular}{c|ccccc|c|c}
      Student & Algorithms & IDK & What should & Go Here & Report & Overall & DIFF\\
      \hline
      Alec F & 0 & 0 & 0 & 0 & 0 & 0 & 0\\
      Dhruv G & 0 & 0 & 0 & 0 & 0 & 0 & 0\\
      Glen T & 0 & 0 & 0 & 0 & 0 & 0 & 0\\
      Sami H & 0 & 0 & 0 & 0 & 0 & 0 & 0\\
      Varun R & 0 & 0 & 0 & 0 & 0 & 0 & 0\\
      
      \hline
      St Dev & 0 & 0 & 0 & 0 & 0 & 0 & 0
    \end{tabular}
  \end{center}
\end{table}

\vspace{0.2in}

\noindent Date: \today.
}
%  -----------------------------------------

%  TODO LIST (COMPLETE THE FULL CHECKLIST - USE AS EXAMPLE THE FIRST CHECKED BOXES!) ----------------------
\newcommand{\addtodo}{
\clearpage
\thispagestyle{empty}

\section*{Read Carefully. Important!}

\noindent By electronically uploading this assignment to Brightspace you acknowledge these statements and accept any repercussions if in any violation of ANY Purdue Academic Misconduct policies. You must upload your homework on time for it to be graded. No late assignments will be accepted. {\bf Only the last uploaded version of your assignment before the due date will be graded}.

\vspace{0.2in}

\noindent {\bf NOTE:} You should aim to submit no later than 30 minutes before the deadline, as there could be last minute network traffic that would cause your assignment to be late, resulting in a grade of zero. 

\vspace{0.2in}

\noindent When submitting your assignment it is assumed that every student considers the below checklist, as there are grading consequences otherwise (e.g., not submitting a cover sheet is an automatic grade of ZERO).

\begin{todolist}

    \item[\done] Your solutions were prepared using the \LaTeX template provided in Brightspace. 
    \item[\done] Your submission has a cover sheet as its first page and this checklist as its second page, according to the template provided.
	 \item[\done] All of your solutions (program code, etc.) are included in the submission as requested. % Check this checkbox and the following ones if satisfied <---
    \item[\done] You have not included any screen shots, photos, etc. (plots should be intermediately saved as .png files and then added into your .tex file). % <---
	 \item[\done] All math notation and algorithms (algorithmic environment) are created using appropriate \LaTeX code (no pictures, handwritten solutions, etc.). % <---
    \item[\done] The .pdf is submitted as an individual file and not in a {\tt .zip}.
    \item[\done] You kept the \LaTeX source code in your files until this assignment is graded, in case you are required to show proof of creating your assignment using \LaTeX.  % <---
    \item[\done] If submitting with a partner, your partner is added in the submission section in Gradescope after you upload your file. % <---
    \item[\done] You have correctly matched each question to its page \# in the .pdf submission in the Gradescope section (after you uploaded your file).
    \item[\done] Watch videos on creating pseudocode if you need a refresher or quick reference to the idea. These are good starter videos:    % <---
    
     \HREF{https://www.youtube.com/watch?v=4jLO0vXPktU}{www.youtube.com/watch?v=4jLO0vXPktU} 
    
    \HREF{https://www.youtube.com/watch?v=yGvfltxHKUU}{www.youtube.com/watch?v=yGvfltxHKUU}
\end{todolist}
}

%% LaTeX
% Für alle, die die Schönheit von Wissenschaft anderen zeigen wollen
% For anyone who wants to show the beauty of science to others

%  -----------------------------------------


\begin{document}


\addcoversheet
\addtodo

\newpage
\tableofcontents
% BEGIN YOUR ASSIGNMENT HERE:

\newpage
\section{Main Text}
\subsection{Introduction}
\subsection{Jacobian-based Saliency Map Attack}
\paragraph{The Jacobian-based Saliency Map Attack is an a greedy algorithm that changes one pixel at a time over many iterations until the modified picture can fool the image classifier model. This type of attack is a good fit for our project because we have a limited budget of pixels, so an algorithm that changes one pixel at a time would be easy to stop when it reaches the pixel budget.}
\subsection{Fast Gradient Signed Method Attack}
\paragraph{ 
The FGSM algorithm begins by loading the necessary libraries, then loading the model using load_model_tf, and then a custom function called random is used to randomly change specific pixel values. 
The main FGSM perturbation function is what is mainly used to make this algorithm attack work. Perturbation is another word for change which is a term used for the FGSM attack. 
It takes four arguments: image, model, epsilon, and target_label. The image is the model's input, the model is the Brightspace model/classifier, epsilon is a scalar value created which controls the magnitude or level of the perturbation, and target_label is the desired output label for the attack. 
The function then calculates the gradient (rate of change/derivative) of the loss with respect to the input image using the TensorFlow GradientTape (a tool used to calculate gradients for the inputted images). 
Then, the perturbation is calculated by multiplying the signed gradient with the epsilon value and then putting that into the original images. 
These original images are provided in the algorithm in two different sections: grass and dandelions. For each section, the images are loaded and then converted into arrays so that they are in the right format to be affected by the FGSM attack. 
Then the perturbed (or changed) image is clipped so that the pixel values are in a valid range of 0 to 1 so that they can be properly processed. 
Finally, the prediction created by the FGSM algorithm previously are made using the Brightspace model and the predictions results are printed or outputted. 
Ultimately, the FGSM attack fools the model by introducing small perturbations (changes) to the input images, making the model misclassify them while keeping the changes imperceptible to humans. 
For my FGSM attack, it fooled the model by having it classify both the grass and dandelion images as grass ([,2] was close to 1 while [,1] was 0 for all the images and [,2] refers to grass probability while [,1] refers to dandelion probability). 

\subsection{PGD Attack}
\paragraph{The Projected Gradient Descent attack is an optimization algorithm that uses white-box attacks in an attempt to perturb images enough so that they can fool an image 
classifier. When researching different types of algorithms to try and work on to fool the pre trained classifier, there were 4 main types of adversarial attacks to consider: poisoning, evasion, extraction, and interference. [1] The most simple, yet feasible option seemed to be an evasion attack because it involves the use of a pre trained model, and it only modifies the input to the model, not the actual training data, unlike the other three options. When looking at the different types of evasion attacks, two stood out: FGSM or Fast Gradient Signed Method and PGD or Projected Gradient Descent. Both involve using a gradient to perturb an image a predetermined amount to add noise and try to fool the classifier. The most simple way of explaining how a gradient based adversarial atack works is that the algorithm computes the gradient of the model's output with respect to its input, and uses it to determine how to perturb the input. FGSM computes the gradient of loss function, a measure of how well the algorithm is performing, with respect to the input and then adds perturbation in the direction of the sign of the gradient. PGD applies FGSM iteratively to the input and projects the perturbed input onto the feasible set of inputs, for example those within a specificed distance of the original input. [2] Since PGD is an interative process, it looks for the perturbation that makes the loss function maximized while ensuring the perturbation is less than the specified input epsilon but that will be discusses in the appendicies.}

\subsection{Conclusion}

\section{Appendix}
\subsection{Testing/Correctness/Verification}
\subsection{Runtime Complexity and Walltime}
\subsection{Performance}
\subsection{Algorithm Justification}

\section{References}
[1] Korolov, Maria. “Adversarial Machine Learning Explained: How Attackers Disrupt AI and ML Systems.” CSO Online, 28 June 2022, www.csoonline.com/article/3664748/adversarial-machine-learning-explained-how-attackers-disrupt-ai-and-ml-systems.html#:~:text=Types%20of%20adversarial%20machine%20learning,evasion%2C%20extraction%2C%20and%20inference. Accessed 27 Apr. 2023.

[2] Knagg, Oscar. “Know Your Enemy - towards Data Science.” Medium, Towards Data Science, 6 Jan. 2019, towardsdatascience.com/know-your-enemy-7f7c5038bdf3. Accessed 27 Apr. 2023.

‌

‌
\end{document}
